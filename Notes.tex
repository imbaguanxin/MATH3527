\documentclass[10pt]{article}  
\usepackage{graphicx}
\usepackage{geometry}   %设置页边距的宏包
\usepackage{algpseudocode}
\usepackage{amsmath, amssymb, amsthm}
\usepackage{enumerate}
\usepackage{enumitem}
\usepackage{framed}
\usepackage{verbatim}
\usepackage{microtype}
\usepackage{kpfonts}
\usepackage{multicol}
\usepackage{amsfonts}
\newcommand{\overbar}[1]{\mkern 1.5mu\overline{\mkern-1.5mu#1\mkern-1.5mu}\mkern 1.5mu}
\newcommand{\Ib}{\mathbf{I}}
\newcommand{\Pb}{\mathbf{P}}
\newcommand{\Qb}{\mathbf{Q}}
\newcommand{\Rb}{\mathbf{R}}
\newcommand{\Nb}{\mathbf{N}}
\newcommand{\Zb}{\mathbf{Z}}
\newcommand{\Zplus}{\mathbf{Z}^+}
\geometry{left=1cm,right=1cm,top=1cm,bottom=1.5cm}  %设置 上、左、下、右 页边距

\begin{document}  
\begin{multicols}{2}
	
\begin{enumerate}
	
	\item Primitive Pythagorean Triple
	\begin{enumerate}
		\item Definition\\
		a triple of numbers(a,b,c) such that a,b,c have no common factors and $a^2 +b^2=c^2$
		\item PPT can be expressed as $a = st, b = \frac{s^2-t^2}{2}, c = \frac{s^2+t^2}{2}, s>t\ge 1 $, gcd(s,t) = 1, s and t are odd.\\
		Thm1: a and b cannot both be odd
		\begin{proof}
			suppose they are all odd\\
			a = 2k+1, b = 2p+1, c = 2z\\
			$a^2+b^2 = 4(k^2+k+p^2+p)+2$, $c^2 = 4z^2$\\
			$a^2+b^2 \equiv 2$ (mod 4), $4|c^2$, contradits.\\
			So a and b cannot both be odd.
		\end{proof}
		Thm2: Suppose a $\leftarrow$ odd b $\leftarrow$ even c $\leftarrow$ odd, $a^2 = c^2 - b^2$, a,b,c are coprime. we want to prove (c-b)(c+b) are coprime.
		\begin{proof}
			Then $a^2 = (c-b)(c+b)$\\
			Suppose they are not coprime(prove by contradict)\\
			Then There exist a prime p s.t. $p|c-b$ or $p|c+b$\\
			$p|(c+b)+(c-b) \Rightarrow p|2c$, $p|(c+b)-(c-b) \Rightarrow p|2b$\\
			Since $p|(c-b)(c+b)\Rightarrow p|a^2 \Rightarrow p|a \Rightarrow$ p is odd.
			Then $p|c \ \& \ p|b$, contradicts with c, b are coprime.\\
			$a^2 = (c-b)(c+b)$ and (c-b), (c+b) are coprime.\\
		\end{proof}				
		Thm3: x,y are coprime, $a^2 =xy\Rightarrow$ x, y are perfect squares. Can be proved by Fundmental Theorem of Arithmetic.\\
		We express $c-b = s^2, \ c+b = t^2$, then $a = st, b = \frac{s^2-t^2}{2}, c = \frac{s^2+t^2}{2}$
	\end{enumerate}

	\item Fermat's Last Theorem\\
	If $n \in \Nb, \ n \ge 3$, $x^n+y^n = z^n$ has no natural number solutions
	
	\item Euclidean algorithm\\
	Suppose $A,B\in \Nb$ There exist unique Q and R such that $Q\in \Nb, R\in \Nb, A = QB+R$. Then gcd(A,B) = gcd(B,R)\\
	Proof of Correctness: 
	\begin{proof}
		Let $d = gcd(A,B), d_0 = gcd(B,R)$\\
		On one hand\\
		$\Rightarrow d|A, d|B \Rightarrow d|R$\\
		$d|B, d|R \Rightarrow d \le d_0$\\
		On the other hand\\
		$d_0|B, d_0|R \Rightarrow d_0 |A \Rightarrow d_0 \le d$\\
		In all: $d = d_0$ i.e. gcd(A,B) = gcd(B,R)
	\end{proof} 

    Thm $LCM(a,b)gcd(a,b) = ab$
	\begin{proof}
		let d = gcd(a,b)\\we need to find LCM(a,b) by find the smallest LCM(a,b) = ja = kb.\\
		Then $j\frac{a}{d} = k\frac{b}{d}$\\
		$\Rightarrow \frac{a}{d} | \frac{b}{d}k$\\
		since gcd($\frac{a}{d},\frac{b}{d}$)=1\\
		$\Rightarrow \frac{a}{d}|k$\\
		smallest k = $\frac{a}{d}$\\
		Same process, we get j = $\frac{b}{d}$\\
		Then LCM = ja = $\frac{ab}{d}\Rightarrow LCM(a,b) \cdot gcd(a,b) = ab$
	\end{proof}

	\item Linear Equations\\
	we can use Euclidean algorithm to get a Linear Equation: $ax_0 + by_0 = gcd(a,b)$. Thus, we can find a solution to $ax + by = n$ iff $gcd(a,b)|n$
	
	Thm1: gcd(m,n) = 1, $m|nc\Rightarrow m|c$
	\begin{proof}
		$\exists x_0,y_0$ s.t. $mx_0+ny_0 = 1$\\
		Then $mx_0c + ny_0c = c$\\
		$m|nc \Rightarrow m|(mx_0c + ny_0c)\Rightarrow m|c$
	\end{proof}
	Thm2: suppose p is prime, $p|ab \Rightarrow p|a$ or $p|b$
	\begin{proof}
		if $p|a$, this is true.
		if $p\nmid a$, then gcd(p,a) = 1 since p is prime.\\
		Then $p|ab \Rightarrow p|b$
	\end{proof}
	Thm3: all solutions to $ax + by = gcd(a,b)$\\
	we can find $ax_0 + by_0 = gcd(a,b)$ by Euclidean algorithm.\\
	Then we have $ax+by = ax_0 + by_0 = gcd(a,b)$\\
	Then $a(x_0-x) + b(y_0-y) = 0\Rightarrow a(x_0 - x) = b(y-y_0)$\\
	Divides both sides by gcd(a,b)\\
	$\frac{a(x_0-x)}{gcd(a,b)} = \frac{b(y-y_0)}{gcd(a,b)}$\\
	$\Rightarrow \frac{a}{gcd(a,b)}|\frac{b}{gcd(a,b)}\cdot (y-y_0)$\\
	 since $\frac{a}{gcd(a,b)},\frac{b}{gcd(a,b)}$ are co-prime $\Rightarrow \frac{a}{gcd(a,b)} | y-y_0$\\
	 $y = y_0 + k \frac{a}{gcd(a,b)}$\\
	 similarly, $x = x_0 - k \frac{b}{gcd(a,b)}$, where k are the same.
	
	\item Fundmental Theorem of Arithmetic\\
	For all $n \in \Nb$ where $n \ge 2$, n factors as a product of prime numbers, and does so in a unique way.

	\item Chapter 8 Congruences\\
	Properties:\\
	$\cdot \ a \equiv a$ (mod m)\\
	$\cdot \ a \equiv b$ (mod m) $\Rightarrow b \equiv a$ (mod m)\\
	$\cdot \ a \equiv b$ (mod m),  $b \equiv c$ (mod m) $\Rightarrow a \equiv c$ (mod m)\\
	$\cdot$ if $a \equiv b$ (mod m),  $c \equiv d$ (mod m) $\Rightarrow$\\
	$a+c \equiv b+d$ (mod m)\\
	$a-c \equiv b-d$ (mod m)\\
	$ac \equiv bd$ (mod m)
	\begin{proof}
		$m|a-b, m|c-d \Rightarrow m|ac-bc, m|bc-bd\Rightarrow\\ m|ac-bc+bc-bd \Rightarrow m|ac-bd\Rightarrow ac\equiv bd$(mod m)
	\end{proof}
	$\cdot$ gcd(m,c) = 1, $ca \equiv cb$ (mod m)$\Rightarrow a \equiv b$ (mod m) 
	\begin{proof}
		$m|ca-cb\Rightarrow m|c(a-d)$ since gcd(m,c) = 1, they are coprime$\Rightarrow m|(a-b)\Rightarrow a\equiv b$(mod m)
	\end{proof}
		
	\item Fermat's Little Theorem\\
	If p is prime, and $p \nmid a$ then $a^{p-1}\equiv 1$ (mod p)
	\item Euler's phi function:\\
	$\phi: \mathbf{N} \rightarrow \mathbf{N}, \phi = \# \{ a | 1 \le a \le m ,gcd(a,m) = 1 \}$\\
	Properties:
	\begin{enumerate}
		\item For prime p: $\phi(p) = p-1$
		\item If gcd(m,n) = 1, $\phi(mn) = \phi(m)\cdot \phi(n)$
		\item For prime p: $\phi(p^k) = p^k - p^{k-1}$
		\item For number $n =p_1^{\alpha_1}p_2^{\alpha_2} \dots p_k^{\alpha_k}$\\
		$\phi(n) = (p_1^{\alpha_1} - p_1^{\alpha_1-1})(p_2^{\alpha_2} - p_2^{\alpha_2-1})\dots (p_k^{\alpha_k} - p_k^{\alpha_k-1})$\\
		$= n(1-\frac{1}{p_1})(1-\frac{1}{p_2})\dots (1-\frac{1}{p_k})$ 
	\end{enumerate}
	
	\item Euler's Phi Formula:\\
	If gcd(a,m) = 1, $a^{\phi(m)}\equiv 1$(mod m)
	\begin{proof}
		Suppose gcd(a,m) = 1. $b_n, 1 \le n \le \phi(m)$ represents all numbers that are co-prime to m.\\
		Consider A = $ab_1, ab_2, ab_3, \dots, ab_{\phi(m)}$ (mod m) and B = $b_1, b_2, b_3,\dots,b_{\phi(m)}$ (mod m). They have the same number of elements. If all elements in A are congruent to different number mod m, two set are the same.\\
		We prove by contradiction, suppose $ab_i \equiv ab_j$ (mod m)$\Rightarrow m | a(b_i - b_j)\Rightarrow m|(b_i - b_j)\Rightarrow b_i \equiv b_j$ contradicts!\\
		Then $b_1b_2\dots b_{\phi(m)} \equiv ab_1ab_2 \dots ab_{\phi(m)}$(mod m)\\ $\Rightarrow \prod_{i = 1}^{\phi(m)}b_i \equiv a^{\phi(m)}\prod_{i = 1}^{\phi(m)}b_i$ (mod m)\\
		since $b_i's$ are coprime to m, $\prod_{i = 1}^{\phi(m)}b_i$ are coprime to m.
		$\Rightarrow 1 \equiv a^{\phi(m)}$ (mod m)
	\end{proof}
	
	\item Chinese Remainder Theorem\\
	If gcd(m,n) = 1, let $b,c \in \Zb$. Then there exist a solution to the simultaneous congruence:
	\begin{equation}
	\left\{
		\begin{array}{lr}
		x \equiv b \ \ mod \ m &\\
		x \equiv c \ \ mod \ m
		\end{array}
	\right.
	\end{equation}
	and such a solution is unique modulo mn
	\begin{proof}
		Existence:\\
		$m|x-b, n|x-c\Rightarrow m\alpha = x-b, m\beta = x - c$\\
		$\Rightarrow m\alpha - n\beta = c-b$ since m,n are coprime\\
		$\Rightarrow$ we can find solution $\alpha_0, \beta_0$ such that $m\alpha_0-n\beta_0 = 1$\\
		$\Rightarrow m\alpha_0(c-b) -n\beta_0(c-b) = c-b$\\ We get $x = m\alpha_0(c-b)+b$\\
		Uniqueness:\\
		suppose that there are two solution $x_0, x_1$\\
		$\Rightarrow x_0 \equiv x_1 \equiv b$ (mod m) $x_0 \equiv x_1 \equiv c$ (mod n)\\
		$\Rightarrow m|(x_1 - x_0), n | (x_1 - x_0) \Rightarrow mn|x_1-x_0$\\
		$\Rightarrow x_1 \equiv x_0$ (mod mn)
	\end{proof}
	\item Solving congruences functions
	\begin{enumerate}
		\item $x^2 \equiv k^2$ (mod p), p is prime\\
		$p|x^2 - k^2 \Rightarrow p|(x-k)(x+k) \\ \Rightarrow x\equiv k$ (mod m) or $x\equiv -k$(mod m)
		\item $a^k \equiv 1$ (mod m), gcd(m,a) = 1\\
		Use Euler's Phi Formula to decrease k.
		\item $ax\equiv c$ (mod m), $gcd(a,m)|c$\\
		There is no solution if $gcd(a,m) \nmid c$\\
		$m|ax - c \Rightarrow ym = ax - c \Rightarrow c = ax - ym$\\
		Find an $x_0$ suits the function by Euclidean Algorithm\\
		Then $ax_0 \equiv c$(mod m). We want to find all x.\\
		Then $ax_0 = ax$ (mod m)\\
		$m | a(x-x_0)\Rightarrow \frac{m}{gcd(m,a)}|\frac{a}{gcd(m,a)}(x-x_0)$\\
		$gcd(\frac{m}{gcd(m,a)},\frac{a}{gcd(m,a)}) = 1\Rightarrow \frac{m}{gcd(m,a)}|x-x_0$\\
		$\Rightarrow x = x_0 + k\frac{m}{gcd(m,a)}$
		\item $x \equiv b$ (mod m), $x \equiv c$ (mod m), gcd(m,n) = 1\\
		Use Chinese Remainder Theorm's proof
	\end{enumerate}
	
	\item Dirichlet's Theorem\\
	If gcd(a,b) = 1, there are infinite number of prime of the form $an+b$.\\
	In book we have primes 3 (mod 4) Theorem, which is a special case of Dirichlet's Theorem.\\
	primes 3 (mod 4) Theorem:\\ There are infinite number of primes of the form $4n+3$
	\begin{proof}
		Prove by Contradiction\\
		Suppose there are finite number of P $\left\{3, p_1, p_2, p_3 ... p_n\right\}$in 4n+3 form.\\
		Consider $A = 4p_1p_2 \dots p_n + 3$\\
		Since A can be factored to product of primes, $A = q_1q_2q_3\dots q_k$\\
		$q_i \equiv 1,3$(mod 4) since primes are odd. At least one of the $q_i \equiv 3$ (mod 4)\\
		If $q_i = 3$, $3| q_1q_2\dots q_k \Rightarrow 3|4p_1p_2\dots p_n + 3 \Rightarrow 3|4p_1p_2\dots p_n $ \\
		Since $3 \nmid 4$, we have $3|p_1p_2 \dots p_n$ which contradicts.($p_i's$ are primes bigger than 3)\\
		If $q_i$ is one of $\left\{p_1,p_2\dots p_n\right\}$\\
		$q_i | A\Rightarrow q_i|4p_1p_2 \dots p_n + 3$\\
		Since $q_i | p_1p_2\dots p_n$, $q_i | 4p_1p_2 \dots p_n$\\
		Then $q_i | 3$. However, $q_i$ is prime and should not divide 3, contradicts.\\
		Therefore, $q_i$ is a new prime of the form $4n+3$.\\
		Therefore, there are infinite number of primes of the form $4n+3$
	\end{proof}

	\item Prime Number Theorem:
	$$\lim_{n\rightarrow \infty}\frac{\pi(n)}{\frac{n}{lnn}} = 1 $$
	where $pi(n):= \#$ of prime numbers $\le$ n
	
	\item Mersenne Primes\\
	Def: primes have the form of $a^n - 1$\\
	Look at geometic series: $S = 1 + x + x^2 +\dots + x^{n-1}$, $x > 1, x\in \Nb$\\
	$S = \frac{x^n-1}{x-1}$. So when $x > 2$, $x^n - 1$ is composite.When $n$ is composite, write $n$ as $pq$. Thus, $x^p-1 | x^{p^q} - 1 = x^n$, i.e $x^n$ is composite.\\
	Thus, Mersenne Primes have the form $2^p - 1$
	
	\item Perfect Numbers\\
	Def: a number equal to the sum of its proper divisors.\\
	$\sigma(n)= \sum_{d \ge 1, d|n}d$  (sum of all the divisors)\\
	$\hat{\sigma}(n) = \sigma(n) - n$, n is perfect iff $\sigma(n) = 2n$\\
	Property of $\sigma(n)$:
	\begin{enumerate}
		\item If P is prime, $\sigma(p^k) = 1 + p + p^2 + \dots + p^k = \frac{p^{k+1}-1}{p-1}$
		\item $gcd(m,n) = 1$, $\sigma(mn) = \sigma(m)\sigma(n)$
	\end{enumerate}
	All even perfect number have the form:$n = 2^{p-1}(2^p-1)$
	\begin{proof}
		n is even, write n as $2^am$, $a \ge 1$, m is odd.\\
		gcd(m, $2^a$) = 1 $\rightarrow \sigma(2^am) = \sigma(2^a)\sigma(m) = (2^{a+1} - 1)\sigma(m)$\\
		since $n$ is perfect number, $\sigma(n) = 2n = 2^{a+1}m$\\
		$\rightarrow 2^{a+1}m = (2^{a+1} - 1)\sigma(m)$\\
		$\rightarrow 2^{a+1}m = (2^{a+1} - 1)(\hat{\sigma}(m) + m)$\\
		$\rightarrow 0 = 2^{a+1}\hat{\sigma}(m) - \hat{\sigma}(m) - m$\\
		$\rightarrow m = (2^{a+1}-1)\hat{\sigma}(m)$\\
		since $a > 1$, $\hat{\sigma}(m)|m, m > \hat{\sigma}(m)$\\
		so $\hat{\sigma}(m)$ is a proper divisor of m and $\hat{\sigma}(m) = \sum d$\\
		$\rightarrow \hat{\sigma}(m) = 1$: m has only one divsor that is smaller than itself: 1$\rightarrow$ m is a prime.\\
		$m = 2^{a+1}-1$, m is a Mersenne Prime. $m$ has the form of $2^p-1$ and $n = 2^{p-1}(2^p-1)$
	\end{proof} 
\end{enumerate}
\newpage
\end{multicols}
\end{document}