\documentclass[10pt]{article}  
\usepackage{graphicx}
\usepackage{geometry}   %设置页边距的宏包

\usepackage{algpseudocode}
\usepackage{amsmath, amssymb, amsthm}
\usepackage{enumerate}
\usepackage{enumitem}
\usepackage{framed}
\usepackage{verbatim}
\usepackage{microtype}
\usepackage{kpfonts}
\usepackage{multicol}
\usepackage{amsfonts}
\newcommand{\overbar}[1]{\mkern 1.5mu\overline{\mkern-1.5mu#1\mkern-1.5mu}\mkern 1.5mu}
\newcommand{\Ib}{\mathbf{I}}
\newcommand{\Pb}{\mathbf{P}}
\newcommand{\Qb}{\mathbf{Q}}
\newcommand{\Rb}{\mathbf{R}}
\newcommand{\Nb}{\mathbf{N}}
\newcommand{\Z}{\mathbf{Z}}
\newcommand{\Zplus}{\mathbf{Z}^+}
\geometry{left=1cm,right=1cm,top=1cm,bottom=1.5cm}  %设置 上、左、下、右 页边距

\begin{document}  
\begin{multicols}{2}
	
\begin{enumerate}
	\item Chapter 8 Congruences\\
	Properties:\\
	$a \equiv a$ (mod m)\\
	$a \equiv b$ (mod m) $\Rightarrow b \equiv a$ (mod m)\\
	$a \equiv b$ (mod m),  $b \equiv c$ (mod m) $\Rightarrow a \equiv c$ (mod m)\\
	if $a \equiv b$ (mod m),  $c \equiv d$ (mod m) $\Rightarrow$\\
	$ a+c \equiv b+d$ (mod m)\\
	$ a-c \equiv b-d$ (mod m)\\
	$ ac \equiv bd$ (mod m)
	\item Chapter 9,10
	\begin{enumerate}
		\item Fermat's Little Theorem\\
		If p is prime, and $p \nmid a$ then $a^{p-1}\equiv 1$ (mod p)
		\item Euler's phi function:\\
		$\phi: \mathbf{N} \rightarrow \mathbf{N}, \phi = \# \{ a | 1 \le a \le m ,gcd(a,m) = 1 \}$\\
		For primes: $\phi(p) = p-1$
		\item Euler's Phi Formula:\\
		If gcd(a,m) = 1, $a^{\phi(m)}\equiv 1$(mod m)\\
		Prove:\\
		Suppose gcd(a,m) = 1. $b_n, 1 \le n \le \phi(m)$ represents all numbers that are co-prime to m.\\
		Consider A = $ab_1, ab_2, ab_3, \dots, ab_{\phi(m)}$ (mod m) and B = $b_1, b_2, b_3,\dots,b_{\phi(m)}$ (mod m). They have the same number of elements. If all elements in A are congruent to different number mod m, two set are the same.\\
		We prove by contradiction, suppose $ab_i \equiv ab_j$ (mod m)$\Rightarrow m | a(b_i - b_j)\Rightarrow m|(b_i - b_j)\Rightarrow b_i \equiv b_j$ contradicts!\\
		Then $b_1b_2\dots b_{\phi(m)} \equiv ab_1ab_2 \dots ab_{\phi(m)}$(mod m)\\ $\Rightarrow \prod_{i = 1}^{\phi(m)}b_i \equiv a^{\phi(m)}\prod_{i = 1}^{\phi(m)}b_i$ (mod m)\\
		since $b_i's$ are coprime to m, $\prod_{i = 1}^{\phi(m)}b_i$ are coprime to m.
		$\Rightarrow 1 \equiv a^{\phi(m)}$ (mod m)
	\end{enumerate}
	
	
	
\end{enumerate}
\newpage
\end{multicols}
\end{document}